\documentclass{article}
\usepackage{amsmath}
\usepackage{amsfonts}
\usepackage[utf8]{inputenc}
\usepackage[a4paper]{geometry}
\usepackage[shortlabels]{enumitem}
\usepackage{hyperref}
\makeatletter
\renewcommand{\@seccntformat}[1]{}
\makeatother
\makeatletter
\newcommand*{\rom}[1]{\expandafter\@slowromancap\romannumeral #1@}
\makeatother
\newcommand{\ul}[1]{\underline{#1}}
\setlength\parindent{0pt}
\title{Mechanik Hausübung Serie 4}
\author{Ahmed Bajra}
\date{2020-10-09}
\begin{document}
	\maketitle
	
	\section{Aufgabe 1}
	Ein Würfel hat im Punkt $G$ die Kinemate $\ul{v}_G= \begin{pmatrix}0\\ v \\2v\end{pmatrix}$, $\ul{\omega}=\begin{pmatrix}\frac{v}a\\\frac{v}a\\-\frac{v}a\end{pmatrix}$
	
	Berechnen Sie die Kinemate im Punkt $D$ und im Punkt $A$.

	\section{Lösung}

	\begin{align*}
		\ul{v}_G=&\ul{v}_D + \ul{\omega}\times\ul{DG}\\
		\ul{DG}=&\begin{pmatrix}-a\\-a\\a\end{pmatrix}\\
		\Rightarrow \ul{v}_G =& \ul{v}_D + \ul0 = \ul{v}_D\\\\
		\ul{v}_G =& \ul{v}_A + \ul\omega\times\ul{AG}\\
		=& \ul{v}_A + \begin{pmatrix}v\\-v\\0\end{pmatrix}\\
		\Leftrightarrow \ul{v}_A =& \ul{v}_G - \begin{pmatrix}v\\-v\\0\end{pmatrix}\\
		=& \begin{pmatrix}-v\\2v\\2v\end{pmatrix}
	\end{align*}

	\section{Aufgabe 2}
	Ein Wurfel führt eine Rotation aus. In der gezeichneten speziellen Lage des 
	Würfels fallen die Kanten $\ul{AB}, \ul{AH}$ und $\ul{AG}$ mit den Achsen $\ul{x}, \ul{y}$ bzw. $\ul{z}$ 
	des Koordinatensystemszusammen. In dieser Lage sind die Geschwindigkeiten der Punkte C bzw. F in kartesischen Komponenten

\end{document}