
\documentclass{article}

\usepackage{amsmath}
\usepackage{amsfonts}
\usepackage[utf8]{inputenc}
\usepackage[a4paper]{geometry}
\usepackage[shortlabels]{enumitem}

\makeatletter
\renewcommand{\@seccntformat}[1]{}
\makeatother

\makeatletter
\newcommand*{\rom}[1]{\expandafter\@slowromancap\romannumeral #1@}
\makeatother

\setlength\parindent{0pt}

\title{Linalg Übung 2}
\author{Ahmed Bajra}
\date{2020-09-25}

\begin{document}

    \maketitle

    \section{Lineare Gleichungssysteme (LGS)}
    Beispiel
    \begin{align*}
        x + 3y &= 11 \\
        4x + 5y &= 23 \\\\
        x = 2&, y = 3
    \end{align*}
    
    Menge aller Lösungen: Lösungsmenge $\mathbb{L}$

    \section{LGS A und B sind \underline{äquivalent}, wenn sie exakt dieselbe Lösungsmenge haben}

    \section{Matrix}

    \begin{align*}
        &a_{11}x_1 + a_{12}x_2 + \dots a_{1n}x_n = b_1 \\ 
        &a_{21}x_1 + a_{22}x_2 + \dots a_{2n}x_n = b_2 \\
        &\dots \\ 
        &a_{m1}x_1 + a_{m2}x_2 + \dots a_{mn}x_n = b_m \\ 
    \end{align*}

    \begin{align*}
        \begin{bmatrix}
            a_{11} & \dotsb & a_{1n} \\
            \dotsb&\dotsb&\dotsb \\
            a_{m1} & \dotsb & a_{mn}
        \end{bmatrix} 
        \cdot
        \begin{bmatrix}
            x_1 \\ 
            \dotsb \\ 
            x_n
        \end{bmatrix} 
        = 
        \begin{bmatrix}
            b_1 \\
            \dotsb \\
            b_m
        \end{bmatrix}
    \end{align*}

    \newpage

    \section{Gaussverfahren}
    \begin{enumerate}[I)]
        \item Vertauschen von Gleichungen
        \item Addieren eines Vielfachen einer Gleichung zu einer anderen
    \end{enumerate}

    \subsection{Kochrezept}

    \begin{enumerate}[1)]
        \item LGS auf Dreiecksform bringen mittels Operationen I) und II)
        \item Rückwärtseinsetzen uum Lösung zu finden
    \end{enumerate}

    \begin{align*}
        3x_1 + 18x_2 + 7x_3 &= 69 \\
        x_1 + 4x_4 + 2x_3 &= 18 \\
        2x_1 + 14x_2 + 8x_3 &= 60
    \end{align*}

    \begin{align*}
        \rightarrow 
        \begin{bmatrix}
            3 & 18 & 7 \\
            1 & 4 & 2 \\
            2 & 14 & 8
        \end{bmatrix}
        \begin{bmatrix}
            x_1 \\ x_2 \\ x_3
        \end{bmatrix}
        = 
        \begin{bmatrix}
            69 \\ 18 \\ 60
        \end{bmatrix}
    \end{align*}

    I) auf 1. und 2. Gleichung anwenden

    
    \begin{align*}
        \rightarrow 
        \begin{bmatrix}
            1 & 4 & 2 \\
            3 & 18 & 7 \\
            2 & 14 & 8
        \end{bmatrix}
        \begin{bmatrix}
            x_1 \\ x_2 \\ x_3
        \end{bmatrix}
        = 
        \begin{bmatrix}
            18 \\ 69 \\ 60
        \end{bmatrix}
    \end{align*}

    II) (2.) - 3 (1); (3.) + 2 (1.)

    II) (3.) - (2) 

    $\rightarrow$

    \begin{align*}
        &\begin{bmatrix}
           1 & 4 & 2 &|& 18 \\
            0 & 6 & 1 &|& 15 \\
            0 & 0 & 3 &|& 9
        \end{bmatrix} \\&\text{Dreiecksform}\\
        &\Rightarrow x_3 = 3 \\
        &\Rightarrow x_2 = 2 \\
        &\Rightarrow x_1 = 4
    \end{align*}

    \newpage
    \section{Tipps Serie 1}

    \begin{enumerate}[(1)]
        \item MC-Aufgabe auf "echo" lösen
        \item Immer gleiche Matrix, verschiedene $\vec{b}, \rightarrow$ Nach $\vec{b}$ lösen
        \subitem alternativ: Auf Dreiecksform Reduzieren, dann nach Aufgaben lösen
        \item Gauss 
        \item Nicht nötig
        \item Nicht nötig
    \end{enumerate}
    
\end{document}