
\documentclass{article}

\usepackage{amsmath}
\usepackage[utf8]{inputenc}
\usepackage[a4paper]{geometry}
\usepackage[shortlabels]{enumitem}

\makeatletter
\renewcommand{\@seccntformat}[1]{}
\makeatother

\makeatletter
\newcommand*{\rom}[1]{\expandafter\@slowromancap\romannumeral #1@}
\makeatother

\setlength\parindent{0pt}

\title{Serie 1}
\author{Ahmed Bajra}
\date{2020-09-20}

\begin{document}

    \maketitle

    \section{Bonusaufaben}
    \subsection{Aufgabe 1}
    \subsubsection{Überlegungen}

    Gegeben sei ein Lineares Gleichungssystem
    \begin{align*}
        a_1 x_1 + a_2 x_2& = a_3 \\
        b_1 x_1 + b_2 x_2& = b_3
    \end{align*}

    Dann gilt: 

    \begin{equation}
        \begin{gathered}
            \\
            \text{Gleichungssystem hat beliebig viele Lösungen wenn: } \\
            \begin{bmatrix}
                a_1 \\
                a_2 \\
                a_3
            \end{bmatrix}
            = c \cdot
            \begin{bmatrix}
                b_1 \\
                b_2 \\
                b_3
            \end{bmatrix} \\
            \text{Da dann die erste Gleichung mittels der Zweiten ausgedrückt werden kann}\\
            \text{und es darum effektiv nur eine Gleichung mit zwei Unbekannten gibt}
            \\
        \end{gathered}
    \end{equation}


    \begin{equation}
        \begin{gathered}
            \\
            \text{Gleichungssystem hat keine Lösung wenn: } \\
            \begin{bmatrix} a_i \\ a_j \end{bmatrix}
            = c \cdot \begin{bmatrix} b_i \\ b_j \end{bmatrix}, 
            a_k \neq cb_k \\
            \text{Da dadurch Gleichungen der folgenden Art resultieren: } \\
            a_k = b_k (\text{obere Bedingung ist aber }  a_k \neq cb_k) \\
            a_k x_k = b_k x_k (\text{hat nur Lösungen, wenn } x_k=0)\\
            \\
        \end{gathered}
    \end{equation}

    \begin{equation}
        \begin{gathered}
            \\
            \text{Gleichungssystem hat genau eine Lösung, wenn weder } (1) \text{ noch } (2) \text{ der Fall ist}
            \\ \\
        \end{gathered}
    \end{equation}

    \newpage

    \subsubsection{Resultate}

    \begin{enumerate}[(a)]
    \item $(2) \Rightarrow $ keine Lösung
    \item $(1) \Rightarrow $ beliebig viele Lösungen
    \item $(2) \Rightarrow \begin{cases} \text{eine Lösung für }x_1,& \text{wenn } x_2=0 \\ \text{keine Lösung }& \text{sonst} \end{cases}$ 
    \item $(1) \Rightarrow $ beliebig viele Lösungen
    \item Eine Lösung, da die zweite Gleichung eine Lösung für $x_3$ liefert und für die beiden anderen Gleichungen (3) gilt
    \item $(1) \Rightarrow $ beliebig viele Lösungen
    \item $(2) \Rightarrow $ keine Lösung (Vergleich: Koeffizienten der 1. und 3. Gleichung)
    \end{enumerate}

    \subsection{Aufgabe 2}
    (f) und (g) haben die jeweils linken Teile der Gleichungen gemeinsam und unterscheiden sich in den Rechten. 

    Da bei (f) alles $=0$ gesetzt wird und die Koeffizienten wie nach (1) ein vielfaches voneinander sind, ergibt sich aus den drei Gleichungen immer dieselbe. 
    Für drei Unbekannte ist diese eine Gleichung nicht ausreichend, weshalb daraus beliebig viele Lösungen zustande kommen.

    Bei (g) würden die ersten zwei Gleichungen dasselbe wie bei (f) bedeuten, doch die dritte ist wiedersprüchlich:

    \begin{align*}
        &(\rom{1}): & 2x_1 + x_2 + x_3& = 2 \\
        &3 \cdot (I) \rightarrow & 3(2x_1 + x_2 + x_3)& = 3\cdot2 \\
        & =& 6x_1 + 3x_2 + 3x_3& = 6 \\ \\
        &(\rom{3}): & 6x_1 + 3x_2 + 3x_3& = 8 \neq 6 \\
    \end{align*}

    \newpage

    \section{Aufgaben zum Rang}

    \subsection{Aufgabe 1}

    Auf das gegebene LGS treffen folgende Aussagen zu:

    \begin{itemize}
        \item (a)
        \item (c)
        \item (d)
    \end{itemize}

    \subsection{Aufgabe 2}

    \begin{align*}
        &(I):& x_1 &+& x_2 &+& 2x_3 &+& 2x_4 &=& b_1 \\
        &(II):& x_1 &+& 2x_2 &+& 3x_3 &+& 4x_4 &=& b_2 \\
        &(III):& x_1 &+& 3x_2 &+& 6x_3 &+& 10x_4 &=& b_3 \\
        &(IV):& x_1 &+& 4x_2 &+& 10x_3 &+& 20x_4 &=& b_4 \\ \\
        &(V) = (II) - (I):&&& x_2 &+& x_3 &+& 2x_4 &=& b_2 - b_1 \\ 
        &(VI) = (III) - (II) - (V):&&&&& 3x_3 &+& 4x_4 &=& b_3 - 2b_2 + b_1 \\
        &(VII) = (IV) - (III) - (V) - (VI):&&&&&&& 4x_4 &=& b_4 - 2b_3 + b_2 
    \end{align*}

    \begin{align*}
        &\Leftrightarrow& x_4 &= \frac14 b_4 - \frac12 b_3 + \frac14 b_2 \\
        &\Leftrightarrow& x_3 &= -\frac13 b_4 - b_3 - b_2 + \frac13 b_1 \\
        &\Leftrightarrow& x_2 &= -\frac16 b_4 + 2b_3 + \frac52 b_2 - \frac43 b_1 \\
        &\Leftrightarrow& x_1 &= -\frac13 b_4 + b_3 + \frac53 b_1 \\
    \end{align*}

    \begin{enumerate}[a)]
        \item $x_1 = 3, x_2 = \frac{22}{3} , x_3 = \frac{16}{3}, x_4 = \frac14$
        \item $x_1 = \frac53, x_2 = \frac{53}{6}, x_3 = \frac23, x_4 = -\frac32$
        \item $x_1 = \frac73, x_2 = 3, x_3 = -2, x_4 = 0$
    \end{enumerate}

    % \newpage

    \subsection{Aufgabe 3}

    \begin{align*}
        &(I):& 3x_1 &+& 4x_2 &+& 2x_3 &=& 8 \\
        &(II):& x_1 &+& 3x_2 &-& x_3 &=& 2 \\\\
        &(I) - 3(II) =&&& -5x_2 &+& 5x_3 &=& 2 \\
        &\Leftrightarrow &&& x_2 && &=& x_3 - \frac25 \\
        &\Leftrightarrow& x_1 &&&&&=& x_3 - \frac25
    \end{align*}

\end{document}