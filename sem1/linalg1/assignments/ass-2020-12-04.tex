\documentclass{article}
\usepackage{amsmath}
\usepackage{amsfonts}
\usepackage{amssymb}
\usepackage[utf8]{inputenc}
\usepackage[a4paper]{geometry}
\usepackage[shortlabels]{enumitem}
\usepackage{hyperref}
\usepackage{tikz}
\makeatletter
\renewcommand{\@seccntformat}[1]{}
\makeatother
\makeatletter
\newcommand*{\rom}[1]{\expandafter\@slowromancap\romannumeral #1@}
\makeatother
\makeatletter
\newcommand*{\mtx}[1]{\begin{pmatrix}#1\end{pmatrix}}
\makeatother
\setlength\parindent{0pt}
\title{Bonusaufgaben 5}
\author{Ahmed Bajra}
\date{2020-11-29}
\begin{document}
	\maketitle
	\section{Aufgabe (a)}
	\label{Aufgabe_(a)}
	\begin{align*}
		\vec{a} =& \mtx{2\\4} \\
		\vec{b} =& \mtx{5\\8}
	\end{align*}
	\begin{align*}
		A:& \vec{a} = \mtx{2\\0} + \frac43\mtx{0\\3} \\
		  & \vec{b} = \frac52\mtx{2\\0} + \frac83\mtx{0\\3}
	\end{align*}

	
	\begin{center}\begin{tikzpicture}[scale=0.3]
		\draw[<->, thin] (-10, 0) -- (10, 0);
		\draw[<->, thin] (0, -10) -- (0, 10);
		\node[red] at (1,3) {$\vec{a}$};
		\draw[->, red, very thick] (0, 0) -- (2,4);
		\node[blue] at (3.5,7) {$\vec{b}$};
		\draw[->, blue, very thick] (0,0) -- (5,8);

		\draw[->, very thick] (0,0) -- (2,0) -- (2,4);
		\draw[->, very thick] (0,0) -- (5,0) -- (5,8);
	\end{tikzpicture}\end{center}

	\newpage

	\begin{align*}
		B:& \vec{a} = 2\mtx{1\\0} + 4\mtx{0\\1}\\
		  & \vec{b} = 5\mtx{1\\0} + 8\mtx{0\\1}
	\end{align*}

	
	\begin{center}\begin{tikzpicture}[scale=0.3]
		\draw[<->, thin] (-10, 0) -- (10, 0);
		\draw[<->, thin] (0, -10) -- (0, 10);
		\node[red] at (0.5,3) {$\vec{a}$};
		\draw[->, red, very thick] (0, 0) -- (2,4);
		\node[blue] at (3.5,7) {$\vec{b}$};
		\draw[->, blue, very thick] (0,0) -- (5,8);
		
		\draw[->, very thick] (0,0) -- (2,0) -- (2,4);
		\draw[->, very thick] (0,0) -- (5,0) -- (5,8);
	\end{tikzpicture}\end{center}

	\begin{align*}
		C:& \vec{a} = 2\mtx{1\\2}\\
		  & \vec{b} \text{ keine mögliche Linearkombination}
	\end{align*}

	\begin{center}\begin{tikzpicture}[scale=0.3]
		\draw[<->, thin] (-10, 0) -- (10, 0);
		\draw[<->, thin] (0, -10) -- (0, 10);
		\node[red] at (2,3) {$\vec{a}$};
		\draw[->, red, very thick] (0, 0) -- (2,4);
		
		\draw[->, very thick] (0,0)-- (1,2);
		\draw[->, very thick] (0,0)-- (2,4);
	\end{tikzpicture}\end{center}

	\newpage

	\begin{align*}
		D:& \vec{a} = 2\mtx{1\\2} \text{ (siehe C)}\\
		  & \vec{b} = \frac52\mtx{1\\2} + \frac12\mtx{5\\6}
	\end{align*}

	\begin{center}\begin{tikzpicture}[scale=0.3]
		\draw[<->, thin] (-10, 0) -- (10, 0);
		\draw[<->, thin] (0, -10) -- (0, 10);
		\node[blue] at (3.5,4.5) {$\vec{b}$};
		\draw[->, blue, very thick] (0,0) -- (5,8);
	
		\draw[->, very thick] (0,0) -- (2.5,5);
		\draw[->, very thick] (2.5,5) -- (5,8);
	\end{tikzpicture}\end{center}

	\begin{align*}
		E:& \vec{a} = \frac12\mtx{4\\8}\\
		  & \vec{b} \text{ keine mögliche Linearkombination}
	\end{align*}

	\begin{center}\begin{tikzpicture}[scale=0.3]
		\draw[<->, thin] (-10, 0) -- (10, 0);
		\draw[<->, thin] (0, -10) -- (0, 10);
		\node[red] at (1,3) {$\vec{a}$};
		\draw[->, red, very thick] (0, 0) -- (2,4);
		
		\draw[->, very thick] (0,0) -- (4,8);
		\draw[->, very thick] (0,0) -- (2,4);
	\end{tikzpicture}\end{center}

	\newpage

	\subsection{Fazit}
	Um einen Vektor $\vec{v} \in \mathbb{R}^2$ mit einer Linearkombination $L: \{\vec{v}_1, ...\vec{v}_n\} \mapsto a_1\vec{v}_1 + ... + a_n\vec{v}_n$ von Vektoren 
	aus einer Menge $M$ abzubilden ($L(M)=\vec{v}$), muss entweder mindestens ein Vektor $\vec{u} \in M$ 
	kollinear zu $\vec{v}$ sein oder es zwei Vektoren $\vec{u}, \vec{w} \in M$ linear 
	unabhängig zueinander sein, es gilt also:

	\begin{align*}
		\exists L: \{\vec{v}_1, ...\vec{v}_n\} \mapsto a_1\vec{v}_1 + ... + a_n\vec{v}_n, M\subset\mathbb{R}^2&: L(M)=\vec{v} \\
		\Leftrightarrow \exists\vec{v}_i\in M: k\vec{v}_i = \vec{v} 
		\text{ oder } \exists\{\vec{v}_i, \vec{v}_j\} \subset M\backslash\{\vec0\}&: k\vec{v}_i\neq\vec{v}_j
	\end{align*}

	\section{Aufgabe (b)}
	
	Da es sich hier stehts um Polynome ersten Grades handelt, kann die Form $p(t)=at+b$ zu einem Vektor $\vec{p}=\mtx{a\\b}$ umgeschrieben werden.

	Die für Linearkombinationen relevanten Operationen sind in beiden Formen analog:

	\begin{align*}
		p_1(t)=a_1t+b_1, & p_2(t)=a_2t+b_2\\
		\vec{p}_1=\mtx{a_1\\b_1}, & \vec{p}_2=\mtx{a_2\\b_2}\\
		\text{Addition: }& p_1(t)+p_2(t) = (a_1+a_2)t + (b_1+b_2)\\
						 & \vec{p}_1+\vec{p}_2 = \mtx{a_1+a_2 \\ b_1+b_2} \rightarrow \text{ entspricht: } (a_1+a_2)t + (b_1+b_2)\\
		\text{Multiplikation mit skalarem Wert: }& kp_1(t) = ka_1t+k_b1\\
												 & k\vec{p}_1 = \mtx{ka_1\\kb_1} \rightarrow \text{ entspricht: } ka_1t+kb_1
	\end{align*}
	
	Übersetzt man alle gegebenen Polynome in die oben genannte Vektordarstellung, kommt man zu denselben Lösungen wie bei \nameref{Aufgabe_(a)}.

\end{document}