
\documentclass{article}

\usepackage{amsmath}
\usepackage{amsfonts}
\usepackage[utf8]{inputenc}
\usepackage[a4paper]{geometry}
\usepackage[shortlabels]{enumitem}
\usepackage{hyperref}

\makeatletter
\renewcommand{\@seccntformat}[1]{}
\makeatother

\makeatletter
\newcommand*{\rom}[1]{\expandafter\@slowromancap\romannumeral #1@}
\makeatother

\setlength\parindent{0pt}

\title{Analysis Übung 1}
\author{Ahmed Bajra}
\date{2020-09-25}


\begin{document}
    \maketitle

    \section{Allgemeine Infos}
    email: melaniol@student.ethz.ch

    Alle zwei Wochen Schnellübung: Freitag 16:15 - 17:45

    \subsection{Serieabgabe}
    \begin{itemize}
        \item Abgabe bis Mittwoch 12:00
        \item bis zu zwei Aufgaben werden korrigiert
        \item Aufgaben markieren die korrigiert werden sollen
        \item Schöne Darstellung
    \end{itemize}

    \subsection{Zwischenprüfung}
    \begin{itemize}
        \item Anfang Frühjahrssemester
        \item Alle Fragen sind MC
        \item Teilnahme ist freiwillig
        \item Wird Notenbonus von 20\% geben (nur aufwertend)
    \end{itemize}

    \subsection{Basisprüfung}
    Hilfsmittel:
    \begin{itemize}
        \item handgeschrieben und vermutlich 10 Seiten
        \item Auf \url{amiv.ethz.ch} alte Zusammenfassung suchen
        \item Alle wichtigen Formeln sind auf Theorieblättern
        \item Formelbuch auch erlaubt (Passerelle Formelbuch :))
        \item Kein Taschenrechner
    \end{itemize}

    \subsection{Ablauf}
    \begin{itemize}
        \item Besprechung vorherige Serie
        \item Kurze Einführung neues Thema
        \item Besprechung neue Theorie
        \item Tipps für neue Serie
    \end{itemize}

    Schnellübungen:

    \begin{itemize}
        \item Hinweise für Serie
        \item Selbstständiges Lösen
        \item Besprechnug schwerste Aufgaben (letzte 15min)
    \end{itemize}

    Vorschläge willkommen

    Theorieblatt der Übung: \url{n.ethz.ch/~melaniol}

    \subsection{Tipps}

    Fragen über Fragen
    \begin{itemize}
        \item Alles Freiwillig
        \item Aufzeichnungen statt live Vorlesung
        \item Freiwillige Serien abgeben nicht nötig
        \item Lernorte
    \end{itemize}

    Weiteres:
    \begin{itemize}
        \item Zusammenfassung stets verwenden
        \item Prüfungsbedingungen verinnerlichen
        \item Serien lösen
        \item Nicht zu lange an einer Aufgabe
        \item Überblick über alle Fächer beihalten
        \item Basisprüfung ist noch weit weg
    \end{itemize}

    \newpage

    \section{Übung}

    \subsection{Bsp 1)}
    Berechne die rekursive und explizite Darstellung der Folge aller ungeraden Zahlen, beginnend bei 3

    Folge: $3, 5, 7, 9\dots$

    Explizite Darstellung: 
    \begin{gather*}
        a_1 = 3 = 2n+1 = 3n = 4n-1 \\
        a_2 = 5 = 2n+1 \neq 3n \neq 4n-1 \\
        \Rightarrow a_n = 2n+1, n= 1,2\dots
    \end{gather*}

    Rekursive Darstellung:

    Explizite Darstellung ist bekannt:
    \begin{align*}
        a_n &= \textbf{2n+1} \\
        a_{n+1} &= 2(n+1) +1\\
        &= \textbf{(2n+1)} +2 \\
        &= a_n+2\\
        \Rightarrow a{n+1} &=a_n+2
    \end{align*}

    \subsection{Eigenschaften}
    \subsubsection{Monotonie}

\end{document}