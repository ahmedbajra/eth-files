
\documentclass{article}

\usepackage{amsmath}
\usepackage{amsfonts}
\usepackage[utf8]{inputenc}
\usepackage[a4paper]{geometry}
\usepackage[shortlabels]{enumitem}
\usepackage{hyperref}

\makeatletter
\renewcommand{\@seccntformat}[1]{}
\makeatother

\makeatletter
\newcommand*{\rom}[1]{\expandafter\@slowromancap\romannumeral #1@}
\makeatother

\setlength\parindent{0pt}

\title{Analysis Übung 3}
\author{Ahmed Bajra}
\date{2020-10-09}

\begin{document}
    \maketitle

    \section{Übungen}
    A6e) \[\lim_{x\to\frac\pi2}(\frac\pi2-x)\tan x \]

    \begin{align*}
        \tan x &= \frac{\sin x}{\cos x} \\
        \lim_{x\to0}\frac{\sin (ax)}{x} &= a\\
        \lim_{x\to\frac\pi2}\frac{\frac\pi2 -x}{\cos x}&\sin x \\
        y=\frac\pi2-x \rightarrow \lim_{x\to\frac\pi2}y&=\lim_{x \to \frac\pi2}\frac\pi2 - x = 0^+ \\
        \Rightarrow x &=\frac\pi2 - y \\
        \Rightarrow \lim_{y\to0^+}\frac{\frac\pi2-\frac\pi2 + y}{\cos (\frac\pi2 - y)}\sin (\frac\pi2-y)&=\lim_{y\to0^+}\frac{y}{\sin}\sin (\frac\pi2 - y)\\
        = 1\cdot1&=1
    \end{align*}

    A6f)

    \begin{align*}
        l &= \lim_{x\to+\infty}\frac{x}2\left(\frac1{\cos\frac\pi{x}}\right)^2\sin\frac{2\pi}{x}\\
        &\text{sub: }y:=\frac1x \\
        &\lim_{x\to+\infty}y=0^+\\
        \Rightarrow l &= \lim_{y\to 0^+}\frac1{2y}\left(\frac1{cos (\pi y)}\right)^2 sin(2\pi y)\\
        &\lim_{y\to 0^+}\frac1{\cos (\pi y)} = 1\\
        \Rightarrow l& = \frac12\frac{sin(2\pi y)}{y} = \frac12 2\pi = \pi
    \end{align*}

    \newpage

    




\end{document}